\documentclass{article}
\usepackage{listings}
\usepackage{xcolor}
\usepackage[a4paper, margin=1in]{geometry}
\setlength{\parindent}{0pt}

\title{Week5 Lab1 Report}
\author{111062117, Hsiang-Sheng Huang}

\begin{document}

\maketitle

\section*{What command you use to build your program? build the library?}
\begin{lstlisting}[language=bash, basicstyle=\ttfamily\small, numbers=left, numberstyle=\tiny\color{gray}, stepnumber=1, frame=single]
$ make all
\end{lstlisting}

In the Makefile, it will build the executable and the staic library and the shared library. So after we make, we will have: \texttt{libmerge\_sort.a}, and \texttt{libmerge\_sort.so}, which are the static and shared libraries respectively.

\section*{What command you use to run the executable? environment variables?}
\begin{lstlisting}[language=bash, basicstyle=\ttfamily\small, numbers=left, numberstyle=\tiny\color{gray}, stepnumber=1, frame=single]
$ ./merge_sort_static
$ LD_LIBRARY_PATH=. ./merge_sort_shared.elf
\end{lstlisting}

The first command runs the executable built from the static library. The second command sets the \texttt{LD\_LIBRARY\_PATH} environment variable to the current directory, which allows the dynamic linker to find the shared library when running the executable built from the shared library.

\end{document}
